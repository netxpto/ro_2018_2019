\clearpage

\section{M-QAM Transmitter}

\begin{tcolorbox}	
	\begin{tabular}{p{2.75cm} p{0.2cm} p{10.5cm}} 	
		\textbf{Header File}   &:& m\_qam\_transmitter.h \\
		\textbf{Source File}   &:& m\_qam\_transmitter.cpp \\
        \textbf{Version}       &:& 20180815 (Andr\'e Mendes)\\
	\end{tabular}
\end{tcolorbox}

This block generates a MQAM optical signal. It can also output the binary sequence. A schematic representation of this block is shown in figure \ref{MQAM_transmitter_block_diagram_simple}.

\begin{figure}[h]
	\centering
	\includegraphics[width=0.6\textwidth]{./lib/m_qam_transmitter/figures/MQAM_transmitter_block_diagram_simple}
	\caption{Basic configuration of the MQAM transmitter}\label{MQAM_transmitter_block_diagram_simple}
\end{figure}

\subsection*{Functional description}

This block generates an optical signal (output signal 1 in figure \ref{MQAM_transmitter_block_diagram}). The binary signal generated in the internal block Binary Source (block B1 in figure \ref{MQAM_transmitter_block_diagram}) can be used to perform a Bit Error Rate (BER) measurement and in that sense it works as an extra output signal (output signal 2 in figure \ref{MQAM_transmitter_block_diagram}).

\begin{figure}[h]
	\centering
	\includegraphics[width=\textwidth]{./lib/m_qam_transmitter/figures/MQAM_transmitter_block_diagram}
	\caption{Schematic representation of the block MQAM transmitter.}\label{MQAM_transmitter_block_diagram}
\end{figure}

\subsection*{Input parameters}

This block has a special set of functions that allow the user to change the basic configuration of the transmitter. The list of input parameters, functions used to change them and the values that each one can take are summarized in table \ref{table}.

\begin{table}[h]
\begin{center}
	\begin{tabular}{| m{3cm} | m{6cm} |  m{2cm} | m{4,3cm} | }
		\hline
		\textbf{Input parameters} & \textbf{Function} & Type & \textbf{Accepted values} \\ \hline
		Mode & setMode() & string & PseudoRandom \newline Random \newline DeterministicAppendZeros \newline DeterministicCyclic \\ \hline
		Number of bits generated & setNumberOfBits() & int & Any integer\\ \hline
		Pattern length & setPatternLength() & int & Real number greater than zero\\ \hline
		Number of bits & setNumberOfBits() & long & Integer number greater than zero\\ \hline
		Number of samples per symbol & setNumberOfSamplesPerSymbol() & int & Integer number of the type $2^n$ with n also integer\\ \hline
		Roll of factor & setRollOfFactor() & double & $\in$ [0,1] \\ \hline
		IQ amplitudes & setIqAmplitudes() & Vector of coordinate points in the I-Q plane & \textbf{Example} for a 4-qam mapping: \{ \{ 1.0, 1.0 \}, \{ -1.0, 1.0 \}, \{ -1.0, -1.0 \}, \{ 1.0, -1.0 \} \} \\ \hline
		Output optical power & setOutputOpticalPower() & int & Real number greater than zero\\ \hline
		Save internal signals & setSaveInternalSignals() & bool & True or False\\
		\hline
	\end{tabular}
	\caption{List of input parameters of the block MQAM transmitter} \label{table}
\end{center}
\end{table}

%\begin{itemize}
%	\item setMode(PseudoRandom);
%	\item setBitPeriod(1.0/50e9);
%	\linebreak (double)
%	\item setPatternLength(3);
%	\linebreak (int)
%	\item setNumberOfBits(10000);
%	\linebreak (long)
%	\item setNumberOfSamplesPerSymbol(32);
%	\linebreak (int)
%	\item setRollOffFactor(0.9);
%	\linebreak (double $\in$ [0,1])
%	\item setIqAmplitudes(\{ \{ 1, 1 \}, \{ -1, 1 \}, \{ -1, -1 \}, \{ 1, -1 \} \});
%	\item setOutputOpticalPower\_dBm(0);
%	\item setSaveInternalSignals(true);
%\end{itemize}

\pagebreak

\subsection*{Methods}

MQamTransmitter(vector$<$Signal *$>$ \&inputSignal, vector$<$Signal *$>$ \&outputSignal); (\textbf{constructor})
\bigbreak

void set(int opt);
\bigbreak
void setMode(BinarySourceMode m)
\bigbreak
BinarySourceMode const getMode(void)
\bigbreak
void setProbabilityOfZero(double pZero)
\bigbreak
double const getProbabilityOfZero(void)
\bigbreak
void setBitStream(string bStream)
\bigbreak
string const getBitStream(void)
\bigbreak
void setNumberOfBits(long int nOfBits)
\bigbreak
long int const getNumberOfBits(void)
\bigbreak
void setPatternLength(int pLength)
\bigbreak
int const getPatternLength(void)
\bigbreak
void setBitPeriod(double bPeriod)
\bigbreak
double const getBitPeriod(void)
\bigbreak
void setM(int mValue)
int const getM(void)
\bigbreak
void setIqAmplitudes(vector$<$t\textunderscore iqValues$>$ iqAmplitudesValues)
\bigbreak
vector$<$t\textunderscore iqValues$>$ const getIqAmplitudes(void)
\bigbreak
void setNumberOfSamplesPerSymbol(int n)
\bigbreak
int const getNumberOfSamplesPerSymbol(void)
\bigbreak
void setRollOffFactor(double rOffFactor)
\bigbreak
double const getRollOffFactor(void)
\bigbreak
void setSeeBeginningOfImpulseResponse(bool sBeginningOfImpulseResponse)
\bigbreak
double const getSeeBeginningOfImpulseResponse(void)
\bigbreak
void setOutputOpticalPower(t\textunderscore real outOpticalPower)
\bigbreak
t\textunderscore real const getOutputOpticalPower(void)
\bigbreak
void setOutputOpticalPower\_dBm(t\_real outOpticalPower\_dBm)
\bigbreak
t\_real const getOutputOpticalPower\_dBm(void)
\pagebreak

\subsection*{Output Signals}

\subparagraph*{Number:} 1 optical and 1 binary (optional)

\subparagraph*{Type:} Optical signal

\subsection*{Example}

\subparagraph*{Cardinality of the constellation:}

This block maps the binary input signal as a \textbf{M-QAM}\footnote{M-QAM: Quadrature Amplitude Modulation with a constellation of M points (4-QAM and 16-QAM are supported)} modulation, by default it uses a 4-QAM modulation as shown in figure \ref{TransmitterConstellation}\footnote{The horizontal axis are in nano measures due to Matlab automatic spacing, therefore the constellation is squared}.

\begin{figure}[h]
	\centering
    \includegraphics[width=\textwidth]{./lib/m_qam_transmitter/figures/MQAM_constellation.pdf}
    \caption{4-QAM Constellation generated in the MQAM Mapper block}\label{TransmitterConstellation}
\end{figure}

\subparagraph*{Constellation Coding:}

By default this block generates the constellation in grey coding. This is possible to observe in figure \ref{TransmitterMapper}. The results obtained are represented in table \ref{MapperTable}

\begin{figure}[h]
	\centering
    \includegraphics[clip, trim=0.5cm 9cm 0.5cm 9cm, width=\textwidth]{./lib/m_qam_transmitter/figures/TransmitterMapper.pdf}
    \caption{Impulses generated in the M-QAM Mapper block}\label{TransmitterMapper}
\end{figure}

\begin{table}[h]
\begin{center}
	\begin{tabular}{| c | c | c | c | }
		\hline
		\textbf{Binary Source} & \textbf{Inline Phase} & \textbf{Quadrature Phase} & \textbf{Location in constellation} \\ \hline
        (0,0) & 1 & 1 & Top right \\ \hline
        (0,1) & -1 & 1 & Top left \\ \hline
        (1,1) & -1 & -1 & Bottom left \\ \hline
        (1,0) & 1 & -1 & Bottom right \\ \hline
	\end{tabular}
	\caption{Analysis of MQAM Mapper output} \label{MapperTable}
\end{center}
\end{table}

\subparagraph*{Pulse Shape:}

This block is charged of shaping the different impulses generated by the \textit{M-QAM Mapper} block into a wave-like signal. The data is encoded with the same temporal spacing has the binary source and has a delay of 10 samples, by default the delay is 0.2 nanoseconds as shown in figure \ref{TransmitterPulseShaper}.

\begin{figure}[h]
	\centering
    \includegraphics[clip, trim=0.5cm 9cm 0.5cm 9cm, width=\textwidth]{./lib/m_qam_transmitter/figures/TransmitterPulseShaper.pdf}
    \caption{Wave generated by the impulses in the Pulse Shaper block}\label{TransmitterPulseShaper}
\end{figure}

\subparagraph*{Impact of a finite impulse response:}

The wave generated in figure \ref{TransmitterPulseShaper} is constituted of the sum of several finite impulse responses separated by 1/200 nanoseconds.

The response of a finite impulse can be visualized in figure \ref{TransmitterImpulseResponse}. The delay visualized is of 0.16 nanoseconds, which corresponds to 8 samples delay. The signal already has a internal delay of 2 samples, therefore the true delay is of 0.2 nanoseconds which is the same as analysed in \textit{Pulse Shape}.

\begin{figure}[h]
	\centering
    \includegraphics[clip, trim=0.5cm 9cm 0.5cm 9cm, width=\textwidth]{./lib/m_qam_transmitter/figures/ImpulseResponse.pdf}
    \caption{Response of a single impulse from the Pulse Shaper block}\label{TransmitterImpulseResponse}
\end{figure}

\subparagraph*{Signal Bandwidth:}

The symbol duration of the signal is 0.2 nanoseconds, the symbol rate is $\frac{1}{\textit{Symbol Duration}} = 5$ MHz and the Roll off factor is 10\%. The signal bandwidth can be determined by $\frac{\textit{Symbol Rate}}{\textit{Roll off factor}}$, therefore the signal bandwidth is 50 MHz.

\subsection*{Open Issues}

Can not use setM(int mValue) function from the M-QAM Mapper block to make a 16-QAM constellation in the present build.

\subsection*{Sugestions for future improvement}

Add to the system another block similar to this one in order to generate two optical signals with perpendicular polarizations. This would allow to combine the two optical signals and generate an optical signal with any type of polarization.
