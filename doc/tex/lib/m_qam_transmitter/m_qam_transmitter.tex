\clearpage

\section{M-QAM Transmitter}

\begin{tcolorbox}	
	\begin{tabular}{p{2.75cm} p{0.2cm} p{10.5cm}} 	
		\textbf{Header File}   &:& m\_qam\_transmitter.h \\
		\textbf{Source File}   &:& m\_qam\_transmitter.cpp \\
        \textbf{Version}       &:& 20180815 (\emph{Helena Ribeiro})\\
	\end{tabular}
\end{tcolorbox}

This block generates a MQAM optical signal. It can also output the binary sequence. A schematic representation of this block is shown in figure \ref{MQAM_transmitter_block_diagram_simple}.

\begin{figure}[h]
	\centering
	\includegraphics[width=0.6\textwidth]{./lib/m_qam_transmitter/figures/MQAM_transmitter_block_diagram_simple}
	\caption{Basic configuration of the MQAM transmitter}\label{MQAM_transmitter_block_diagram_simple}
\end{figure}

\subsection*{Functional description}

This block generates an optical signal (output signal 1 in figure \ref{MQAM_transmitter_block_diagram}). The binary signal generated in the internal block Binary Source (block B1 in figure \ref{MQAM_transmitter_block_diagram}) can be used to perform a Bit Error Rate (BER) measurement and in that sense it works as an extra output signal (output signal 2 in figure \ref{MQAM_transmitter_block_diagram}).

\begin{figure}[h]
	\centering
	\includegraphics[width=\textwidth]{./lib/m_qam_transmitter/figures/MQAM_transmitter_block_diagram}
	\caption{Schematic representation of the block MQAM transmitter.}\label{MQAM_transmitter_block_diagram}
\end{figure}

\subsection*{Input parameters}

This block has a special set of functions that allow the user to change the basic configuration of the transmitter. The list of input parameters, functions used to change them and the values that each one can take are summarized in table \ref{table}.

\begin{table}[h]
\begin{center}
	\begin{tabular}{| m{3cm} | m{6cm} |  m{2cm} | m{4,3cm} | }
		\hline
		\textbf{Input parameters} & \textbf{Function} & Type & \textbf{Accepted values} \\ \hline
		Mode & setMode() & string & PseudoRandom \newline Random \newline DeterministicAppendZeros \newline DeterministicCyclic \\ \hline
		Number of bits generated & setNumberOfBits() & int & Any integer\\ \hline
		Pattern length & setPatternLength() & int & Real number greater than zero\\ \hline
		Number of bits & setNumberOfBits() & long & Integer number greater than zero\\ \hline
		Number of samples per symbol & setNumberOfSamplesPerSymbol() & int & Integer number of the type $2^n$ with n also integer\\ \hline
		Roll of factor & setRollOfFactor() & double & $\in$ [0,1] \\ \hline
		IQ amplitudes & setIqAmplitudes() & Vector of coordinate points in the I-Q plane & \textbf{Example} for a 4-qam mapping: \{ \{ 1.0, 1.0 \}, \{ -1.0, 1.0 \}, \{ -1.0, -1.0 \}, \{ 1.0, -1.0 \} \} \\ \hline
		Output optical power & setOutputOpticalPower() & int & Real number greater than zero\\ \hline
		Save internal signals & setSaveInternalSignals() & bool & True or False\\
		\hline
	\end{tabular}
	\caption{List of input parameters of the block MQAM transmitter} \label{table}
\end{center}
\end{table}

%\begin{itemize}
%	\item setMode(PseudoRandom);
%	\item setBitPeriod(1.0/50e9);
%	\linebreak (double)
%	\item setPatternLength(3);
%	\linebreak (int)
%	\item setNumberOfBits(10000);
%	\linebreak (long)
%	\item setNumberOfSamplesPerSymbol(32);
%	\linebreak (int)
%	\item setRollOffFactor(0.9);
%	\linebreak (double $\in$ [0,1])
%	\item setIqAmplitudes(\{ \{ 1, 1 \}, \{ -1, 1 \}, \{ -1, -1 \}, \{ 1, -1 \} \});
%	\item setOutputOpticalPower\_dBm(0);
%	\item setSaveInternalSignals(true);
%\end{itemize}

\pagebreak

\subsection*{Methods}

MQamTransmitter(vector$<$Signal *$>$ \&inputSignal, vector$<$Signal *$>$ \&outputSignal); (\textbf{constructor})
\bigbreak

void set(int opt);
\bigbreak
void setMode(BinarySourceMode m)
\bigbreak
BinarySourceMode const getMode(void)
\bigbreak
void setProbabilityOfZero(double pZero)
\bigbreak
double const getProbabilityOfZero(void)
\bigbreak
void setBitStream(string bStream)
\bigbreak
string const getBitStream(void)
\bigbreak
void setNumberOfBits(long int nOfBits)
\bigbreak
long int const getNumberOfBits(void)
\bigbreak
void setPatternLength(int pLength)
\bigbreak
int const getPatternLength(void)
\bigbreak
void setBitPeriod(double bPeriod)
\bigbreak
double const getBitPeriod(void)
\bigbreak
void setM(int mValue)
int const getM(void)
\bigbreak
void setIqAmplitudes(vector$<$t\textunderscore iqValues$>$ iqAmplitudesValues)
\bigbreak
vector$<$t\textunderscore iqValues$>$ const getIqAmplitudes(void)
\bigbreak
void setNumberOfSamplesPerSymbol(int n)
\bigbreak
int const getNumberOfSamplesPerSymbol(void)
\bigbreak
void setRollOffFactor(double rOffFactor)
\bigbreak
double const getRollOffFactor(void)
\bigbreak
void setSeeBeginningOfImpulseResponse(bool sBeginningOfImpulseResponse)
\bigbreak
double const getSeeBeginningOfImpulseResponse(void)
\bigbreak
void setOutputOpticalPower(t\textunderscore real outOpticalPower)
\bigbreak
t\textunderscore real const getOutputOpticalPower(void)
\bigbreak
void setOutputOpticalPower\_dBm(t\_real outOpticalPower\_dBm)
\bigbreak
t\_real const getOutputOpticalPower\_dBm(void)
\pagebreak

\subsection*{Input Signals}
\subparagraph*{Number:} 2

\subparagraph*{Type:} Binary

\subsection*{Output Signals}

\subparagraph*{Number:} 1 optical and 1 binary (optional)

\subparagraph*{Type:} Optical signal

\subsection*{Example}
\subparagraph*{Cardinality of the constellation}
As the figure \ref{constellation} shows, the constellation used has cardinality 4.

\begin{figure}[hbtp]
\centering
\includegraphics[scale=0.5]{./lib/m_qam_transmitter/figures/MQAM_constellation.png}
\caption{Constellation}\label{constellation}
\end{figure}

\subparagraph*{Constellation coding}
The Gray constellation is used because only one bit is exchanged between variations. This reduces errors.
\subparagraph*{Pulse shape}
The figures \ref{S0} through \ref{S8} are examples of the outputs/inputs of the various transmitter blocks
\begin{figure}[H]
\centering
\includegraphics[scale=0.35]{./lib/m_qam_transmitter/figures/S0.PNG}
\caption{BinaryData}\label{S0}
\end{figure}

\begin{figure}[H]
\centering
\includegraphics[scale=0.35]{./lib/m_qam_transmitter/figures/S1_S2.PNG}
\caption{I(t) and Q(t) signals}\label{S1_S2}
\end{figure}

As shown in the table \ref{valuesMQAM} the MQAM mapper block uses the constellation \ref{constellation} to encode the binary input (Figure \ref{S0}).

\begin{table}[htb]
\caption{Values from the above example}\label{valuesMQAM}
\centering
\begin{tabular}{|l|l|l|l|l|l|l|l|l|l|l|l|l|l|l|l|l|}
\hline
Binary & 01 & 00 & 10 & 10 & 11 & 10 & 01 & 10 & 10 & 00 & 10 & 01 & 11 & 10 & 00 & 10\\ \hline
I & 1 & 1 & -1 & -1 & -1 & -1 & 1 & -1 & -1 & 1 & -1 & 1 & -1 & -1 & 1 & -1 \\ \hline
Q & -1 & 1 & 1 & 1 & -1 & 1 & -1 & 1 & 1 & 1 & 1 & -1 & -1 & 1 & 1 & 1 \\ \hline
\end{tabular}
\end{table}

The time to continuouse block produces a drac \ref{S3_S4} with the same sequence from I(t)and Q(t). 
\begin{figure}[H]
\centering
\includegraphics[scale=0.35]{./lib/m_qam_transmitter/figures/S3_S4.PNG}
\caption{Drac signal}\label{S3_S4}
\end{figure}
The eye diagram of drac's signal \ref{S3_S4_eye} has a big gap meaning a low interference between symbols.
\begin{figure}[H]
\centering
\includegraphics[scale=0.35]{./lib/m_qam_transmitter/figures/S3_S4_eye.PNG}
\caption{Drac signal eye diagram}\label{S3_S4_eye}
\end{figure}
The pulse shaper block produces a impulse response \ref{S5_S6}. In the eye diagram \ref{S5_S6_eye} and \ref{S5_S6_IES} we are able to see a bigger gap comparing to the drac's eye diagram and the IES=0.
\begin{figure}[H]
\centering
\includegraphics[scale=0.35]{./lib/m_qam_transmitter/figures/S5_S6.PNG}
\caption{impulse response}\label{S5_S6}
\end{figure}

\begin{figure}[H]
\centering
\includegraphics[scale=0.35]{./lib/m_qam_transmitter/figures/S5_S6_eye.PNG}
\caption{impulse response eye diagram}\label{S5_S6_eye}
\end{figure}

\begin{figure}[H]
\centering
\includegraphics[scale=0.35]{./lib/m_qam_transmitter/figures/S5_S6_IES.PNG}
\caption{impulse response ies}\label{S5_S6_IES}
\end{figure}
As the figure \ref{S8} shows the output of the transmitter block still has a null IES.
\begin{figure}[H]
\centering
\includegraphics[scale=0.35]{./lib/m_qam_transmitter/figures/S8_eye.PNG}
\caption{Output signal eye diagram}\label{S8}
\end{figure}


\subparagraph*{Impact of a finite impulse response (pulse shaper)}
Pulse shaper block applies a raised-cosine family impulse filter in the input. Using a finite impulse we are able to specify the duration of the response and the frequencies to perturb. This means we can minimize the probability of error.
\subparagraph*{Signal bandwidth}
As the bandwidth of the channel decreases, the interference between symbols increases.
\subparagraph*{Inter-symbolic interference}
IES is null in all the signals of the transmitter block.

%\begin{figure}[h]
%	\centering
%	\includegraphics[width=0.8\textwidth]{./lib/m_qam_transmitter/figures/BinarySource_output}
%	\caption{Example of the binary sequence generated by this block for a sequence 0100...}
%\end{figure}
%
%%\begin{figure}[h]
%	\centering
%	\includegraphics[width=0.8\textwidth]{./lib/m_qam_transmitter/figures/IQmodulator0_output}
%	\caption{Example of the output optical signal generated by this block for a sequence 0100...}
%\end{figure}

\subsection*{Sugestions for future improvement}

Add to the system another block similar to this one in order to generate two optical signals with perpendicular polarizations. This would allow to combine the two optical signals and generate an optical signal with any type of polarization.
